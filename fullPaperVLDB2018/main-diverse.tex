% THIS IS AN EXAMPLE DOCUMENT FOR VLDB 2012
% based on ACM SIGPROC-SP.TEX VERSION 2.7
% Modified by  Gerald Weber <gerald@cs.auckland.ac.nz>
% Removed the requirement to include *bbl file in here. (AhmetSacan, Sep2012)
% Fixed the equation on page 3 to prevent line overflow. (AhmetSacan, Sep2012)

\documentclass{vldb}
\usepackage{graphicx}
\usepackage{balance}  % for  \balance command ON LAST PAGE  (only there!)
\usepackage{url}
\usepackage{makecell}
\usepackage{booktabs} % For formal tables
\usepackage{color,soul}
\usepackage{fancyvrb}
\usepackage{amsmath}
%\usepackage{amsthm}
\usepackage{subcaption}
\captionsetup{compatibility=false}
\usepackage[T1]{fontenc}
\usepackage{dsfont}
\fvset{frame=single,framesep=1mm,fontfamily=tt,numbers=left,framerule=.3mm,numbersep=1mm,commandchars=\\\{\}}
\definecolor{bgred}{RGB}{255,210,205}
\definecolor{bgblue}{RGB}{210,220,255}
\definecolor{bgyellow}{RGB}{240,240,190}
\definecolor{bggrey}{RGB}{223,223,225}
\definecolor{purple}{RGB}{180,0,180}
\usepackage{colonequals}
\usepackage{paralist}
\usepackage{booktabs}
\usepackage{tabularx}
\usepackage{multirow}
\usepackage{xfrac}
\usepackage[boxed,noend]{algorithm2e}

\newtheorem{theorem}{Theorem}[section]
\newtheorem{proposition}[theorem]{Proposition}
\newtheorem{lemma}[theorem]{Lemma}
\newtheorem{corollary}[theorem]{Corollary}
\newtheorem{observation}[theorem]{Observation}
\newtheorem{example}[theorem]{Example}
\newtheorem{definition}[theorem]{Definition}

%\fancyhead{}
%\settopmatter{printacmref=false, printfolios=false}

\newcommand{\REMARK}[2]{\textcolor{red}{{\bf #1: }#2}}
\newcommand{\TODO}[1]{\textcolor{red}{[{\bf TODO: }#1]}}
\newcommand{\tova}[1]{\textcolor{purple}{[{\bf Tova: }#1]}}
\newcommand{\amit}[1]{\textcolor{blue}{[{\bf Amit: }#1]}}
\newcommand{\yael}[1]{\textcolor{red}{[{\bf Yael: }#1]}}
\newcommand{\brit}[1]{\textcolor{green}{[{\bf Brit: }#1]}}
%\newcommand{\tova}[1]{\textcolor{purple}{\REMARK{Yael}{#1}}
%\newcommand{\brit}[1]{\textcolor{green}{\REMARK{Yael}{#1}}
\newcommand{\scream}[1]{\REMARK{\underline{TBD}}{#1}}
%\newcommand{\icde}[1]{\textcolor{DarkRed}{#1}}
\newcommand{\icde}[1]{#1}
\newcommand{\eat}[1]{}
\newcommand{\highlight}[1]{%
	\colorbox{red!20}{$\displaystyle#1$}}
\newcommand{\highlightblue}[1]{%
	\colorbox{blue!20}{$\displaystyle#1$}}

% general macros
\newcommand{\var}[1]{\operatorname{\mathit{#1}}}
\newcommand{\name}[1]{\operatorname{#1}}
\newcommand{\funct}[2]{\operatorname{#1}\!\left({#2}\right)}
\newcommand{\To}{\!\rightarrow\!}
\newcommand{\card}[1]{\left|{#1}\right|}
\newcommand{\ceil}[1]{\left\lceil{#1}\right\rceil}
\renewcommand{\epsilon}{\varepsilon}
\renewcommand{\phi}{\varphi}
\newcommand{\oof}[1]{\operatorname{O}\!\left({#1}\right)}
\newcommand{\oofi}[1]{\operatorname{O}({#1})}
\newcommand{\omegaof}[1]{\operatorname{\Omega}\!\left({#1}\right)}
\newcommand{\omegaofi}[1]{\operatorname{\Omega}({#1})}
\newcommand{\thetaof}[1]{\operatorname{\Theta}\!\left({#1}\right)}
\newcommand{\thetaofi}[1]{\operatorname{\Theta}({#1})}

%typeface
\newcommand{\textcd}[1]{\textup{#1}}
\newcommand{\texterm}[1]{\textup{``#1''}}
\newcommand{\elem}[1]{\textcd{#1}}

% terms
\newcommand{\fset}{fact-set}
\newcommand{\Fset}{Fact-set}
\newcommand{\clss}{\mathcal{E}}
\newcommand{\rels}{\mathcal{R}}
\newcommand{\qlang}{\textup{\textsf{SPARQ-U}}}
\newcommand{\sysname}{\textup{\textsf{PODIUM}}}
\newcommand{\oassis}{\textup{\texttt{OASSIS}}}
\newcommand{\oassisql}{\textup{\texttt{OASSIS-QL}}}
\newcommand{\histories}{\textup{\texttt{activities}}}
%\newcommand{\cic}{\mathcal{C}_{\name{ic}}}
\newcommand{\tax}{\Psi}
\newcommand{\blnk}{\underline{\hspace{2.5mm}}}

%functions
\newcommand{\ic}[1]{\funct{ic}{#1}}
\newcommand{\icsim}[1]{\funct{icsim}{#1}}
\newcommand{\nicsim}[1]{\funct{\tilde{icsim}}{#1}}
\newcommand{\supsim}[1]{\funct{supsim}{#1}}
\newcommand{\fullsim}[1]{\funct{sim}{#1}}
\newcommand{\supp}[2][u]{\funct{supp_{\mathit{#1}}}{#2}}



% Include information below and uncomment for camera ready
\vldbTitle{PODIUM: Procuring Opinions from Diverse Users in a Multi-Dimensional World}
\vldbAuthors{}
\vldbDOI{https://doi.org/TBD}
	
\usepackage[pdfborder={0 0 0}, plainpages, pdfpagelabels=false, pdfstartview=FitH]{hyperref}

\begin{document}

% ****************** TITLE ****************************************

\title{PODIUM: Procuring Opinions from Diverse Users in a Multi-Dimensional World}


% possible, but not really needed or used for PVLDB:
%\subtitle{[Extended Abstract]
%\titlenote{A full version of this paper is available as\textit{Author's Guide to Preparing ACM SIG Proceedings Using \LaTeX$2_\epsilon$\ and BibTeX} at \texttt{www.acm.org/eaddress.htm}}}

% ****************** AUTHORS **************************************

% You need the command \numberofauthors to handle the 'placement
% and alignment' of the authors beneath the title.
%
% For aesthetic reasons, we recommend 'three authors at a time'
% i.e. three 'name/affiliation blocks' be placed beneath the title.
%
% NOTE: You are NOT restricted in how many 'rows' of
% "name/affiliations" may appear. We just ask that you restrict
% the number of 'columns' to three.
%
% Because of the available 'opening page real-estate'
% we ask you to refrain from putting more than six authors
% (two rows with three columns) beneath the article title.
% More than six makes the first-page appear very cluttered indeed.
%
% Use the \alignauthor commands to handle the names
% and affiliations for an 'aesthetic maximum' of six authors.
% Add names, affiliations, addresses for
% the seventh etc. author(s) as the argument for the
% \additionalauthors command.
% These 'additional authors' will be output/set for you
% without further effort on your part as the last section in
% the body of your article BEFORE References or any Appendices.

\numberofauthors{2} %  in this sample file, there are a *total*
% of EIGHT authors. SIX appear on the 'first-page' (for formatting
% reasons) and the remaining two appear in the \additionalauthors section.

\author{
% You can go ahead and credit any number of authors here,
% e.g. one 'row of three' or two rows (consisting of one row of three
% and a second row of one, two or three).
%
% The command \alignauthor (no curly braces needed) should
% precede each author name, affiliation/snail-mail address and
% e-mail address. Additionally, tag each line of
% affiliation/address with \affaddr, and tag the
% e-mail address with \email.
%
% 1st. author
\alignauthor
Yael Amsterdamer\\
       \affaddr{Bar Ilan University}\\
       \email{first.last@biu.ac.il}
% 2nd. author
\alignauthor
Oded Goldreich\\
       \affaddr{Bar Ilan University}\\
       \email{first.last@live.biu.ac.il}      
% 3rd. author
}
% There's nothing stopping you putting the seventh, eighth, etc.
% author on the opening page (as the 'third row') but we ask,
% for aesthetic reasons that you place these 'additional authors'
% in the \additional authors block, viz.

% Just remember to make sure that the TOTAL number of authors
% is the number that will appear on the first page PLUS the
% number that will appear in the \additionalauthors section.


\maketitle

\begin{abstract}
%	Many modern applications, such as crowdsourcing platforms, social networks and recommender systems, include a crucial component that selects user profiles from a repository. The implementation of this component is application-specific and laborious, often depending on semantically rich user information. To assist application developers in this task, we propose a novel declarative framework that includes (1) a query language that extends SPARQL to allow capturing the properties and similarity of (query-defined parts of) user profiles/data, (2) a generic similarity measure allowing to compare (parts of) user profiles and to support soft constraints in a semantically rich environment, and (3) an efficient query evaluation algorithm derived from a theoretical analysis of our novel similarity metric. Our experimental study on real-life data indicates the effectiveness and flexibility of our approach, showing in particular that it outperforms existing task-specific solutions in common user selection tasks.

To Be Written.
\end{abstract}



\section{Introduction}
\label{sec:intro} 

% Something general about the need (focus on crowdsourcing probably)

The need to procure a diverse and representative set of opinions arises in multiple contexts, such as surveys, market research, and crowdsourcing applications. Consider, for example, a traveler planning a trip and looking for specific ``tips'' on some destination; an owner of a new restaurant wishing to perform a preliminary customer survey; or a website manager seeking usability feedback. Platforms such as Yelp (\url{https://www.yelp.com}), that have a large user base and high-dimensional, rich data on each user, provide an opportunity for procuring opinions from a diverse set of users. At the same time, these characteristics of the data also pose challenges in realizing this potential: how do we \emph{define} diversity while accounting for high-dimensional data? Can we \emph{efficiently compute} a diverse subset of users? Can the resulting selection be \emph{explained to} and \emph{customized by} the client user? 

The latter challenge is of particular interest since the requirements on user selection may greatly vary across different scenarios. For instance, a traveler may seek the opinions of users with different culinary preferences, whereas a website manager may seek feedback from users with diverse activity history. 
%The latter challenge reflects a desideratum that the choice of users is not made as a ``black box", since the requirements on user selection may greatly vary across different settings. For instance, the client user may wish to focus only on users who visited a particular restaurant or ones that like its type of cuisine, fixing some characteristics and wishing for diversity in others. 

%While user selection 
%
%Most of the previously proposed crowdsourcing solutions either do not allow for user selection \cite{..} at all, or do not account for diversity \cite{...}. Multiple notions of diversity were proposed, but... \scream{DISCUSS, say that not explainable/configurable/something}. See discussion of related work in Section 2. 
%
%
%
%To account for diversified user selection   




%For example, hotels and restaurants owners often wish to conduct a market research, whose results would allow them to place targeted advertisements; as another example, the effectiveness of crowdsourcing applications \scream{such as..} often requires a representative set of opinions \scream{...}. \scream{Maybe focus on crowdsourcing to begin with}. 


% What existing solutions so and why they do not suffice (briefly+point to related work)


Previous work on diversification either focus on covering a range or a set of categories, and thus do not account for covering the full range of opinions in multiple dimensions, which is provided in user profiles and can be leveraged for user selection. Moreover, explanations and customization  has not been considered in this context. See Section~\ref{sec:related} for details.
%(1) focuses on diversifying users rather than opinions, thus not accounting for the multi-faceted data of their previous interactions with the system, and (2) do not support explanation and customization. See Section 2.4 for details.   

To address these challenges, we introduce \sysname{}:  
a novel tool for the procurement of diverse opinions, utilizing multi-dimensional user profiles. Our solution consists of the following components:

\paragraph*{User Profile Model} The model that we consider for user profiles enables capturing, in a uniform manner, personal characteristics of users 
(e.g., nationality) and their past interactions with the platform (e.g., feedback they provided on restaurants). These properties are associated with a score from some range (Boolean, rating score, etc.) and thus form high-dimensional data.

%High-dimensional information of both kinds is openly available in frameworks such as Tripadvisor: we have retrieved over \scream{how many?} properties of users, where values are e.g. boolean/rating/etc.        

\paragraph*{Capturing Diversity} 
Different notions of diversity has been considered in the literature (see  Section~\ref{sec:related}). In the present work, we focus on a notion of diversity that is \emph{coverage-based}, \emph{customizable} and designed for the multi-dimensional, rich contents of user profiles, as briefly explained next.

\emph{Coverage-based diversity} aims to select a set (of users, in our case) that in some sense represents or ``covers'' many of the different, possibly overlapping \emph{groups} within a source population~\cite{agrawal2009diversifying,servajean2013profile,wu2015hear}.
This class of diversity notions fits typical scenarios of opinion procurement (e.g., surveys, market research) in comparison with \emph{distance-based diversity}, which focuses on maximizing the differences between the members of the selected group~\cite{wu2015hear}. We provide a specific problem definition for coverage-based diversity in our setting, relying on the available properties in user profiles.

\emph{Customizable} diversity allows the client an informed control over the user groups/data dimensions whose coverage is targeted. For that, we define general notions of  \emph{explanations} for the user selection result, which enable visualizing the coverage of different user properties and the role of such properties in the selection of each user. The client can then provide, via a user-friendly interface, a \emph{feedback} with well-defined semantics that serves to refine the user selection.     %For instance, a journalist that seeks diverse opinions related to Mexican food in some region, might wish to choose users that are diverse in every respect other than love for Mexican food. As another example, a website manager seeking diverse opinions on its usability might prefer users that are diverse mainly in their level of activity in the website.



\paragraph*{Complexity and Algorithms} Based on our model, we formalize the problem of optimizing user subset diversity. The corresponding decision problem is NP-complete and thus we employ an effective greedy algorithm with provable approximation guarantees.  

%\paragraph*{Customization and Explanations} An important feature of our solution is that it allows an interactive process where the client user feeds her diversity requirements \scream{such as..}, \sysname computes a proposed selection of users, and graphically shows and explain features of the selection \scream{such as..}. This user-friendly interface allows the client user to understand the selection, and if needed to refine her criteria and repeat the selection process, before the selection is finalized and requests are sent.  







\paragraph*{Experimentation Methods} We will examine our approach using data from large-scale existing datasets in respect to relevant alternative algorithms. We provide several means and measures, some generic and some tailor-made, to try and accurately assess diversification achieved by our method. Aside from performing \emph{qualitative} tests and \emph{performance} benchmark, we will show that our approach is successfully able to predict a selection of users that would yield a diverse set of opinions in a real-world scenario.

\paragraph*{Paper Outline} Section~\ref{sec:prelim} contains the model we use to represent user profiles. In Section~\ref{sec:diversity} we present several possible definitions for diversity as well as a formal definition, and in Section~\ref{sec:coverage} we propose and develop the concept of \emph{Coverage-based} diversity. Algorithmic and computational concerns are described in Section~\ref{sec:compute}, while experiments and results are shown in Section~\ref{sec:Implementation}. Section~\ref{sec:related} consists of related work and we conclude in Section~\ref{sec:conc}.
%\vspace{-2px}
\section{User Profiles}

\label{sec:prelim} 

\begin{table}
	{\small
		\resizebox{.4\textwidth}{!}{%
		\begin{tabularx}{\linewidth}{llllll}
			\toprule
			\textbf{Property} & \textbf{Alice} & \textbf{Bob} & \textbf{Carol} & \textbf{David}  & \textbf{Eve}\\
			\midrule
			livesIn & Tokyo${}^{(2)}$ & NYC${}^{(1)}$ & Bali${}^{(1)}$  &  Paris${}^{(1)}$   & Tokyo  \\
			ageGroup & 50-64${}^{(1)}$ & -- & 25-34${}^{(1)}$ & -- & -- \\ 
			avgRating Mexican & 0.95${}^{(2)}$ & 0.6${}^{(1)}$ & -- & 0.7 & 0.15${}^{(1)}$ \\
			visitFreq Mexican & 0.8${}^{(2)}$ & 0.25${}^{(2)}$ & -- &  0.75& 0.2 \\
			avgRating CheapEats & 0.1${}^{(2)}$ & 0.5${}^{(2)}$ & 0.6 & -- & 0.3\\
			visitFreq CheapEats & 0.6${}^{(1)}$ & 0.85${}^{(2)}$ & 0.9 & --  & 0.2 ${}^{(1)}$ \\
			\bottomrule
		\end{tabularx}
	}
	} 
\caption{Example user profiles}
\label{tab:profs} 
\end{table}

To be able to procure diverse opinions from users, we first need to properly model the profiles of users.    
Let~$\mathcal{U}$ be a population of users and~$\mathcal{P}$ be some domain of properties. Following~\cite{amsterdamer2016december}, we define the profile of a user $u\in \mathcal{U}$ as a tuple $D_u=\langle P_u, S_u \rangle$ where $P_u\subseteq\mathcal{P}$ is a set of properties relevant to $u$ and $S_u:P_u\To [0,1]$ maps each property to a score (normalized to~$[0,1]$). This score may have different interpretations depending on the type of property, e.g., true/false, user rating, and so on, and may be provided directly by $u$ or automatically derived from $u$'s activity in the website.  

\begin{example}
	Table~\ref{tab:profs} shows a few profiles from a travel website (for now, ignore the numbers in superscript). In the first two rows, \elem{livesIn~<city>} ~and~ \elem{ageGroup~<X-Y>} are true/false properties for relevant cities and age ranges. E.g., \elem{livesIn~Tokyo} is a property with score~1 (i.e., true) in Alice's profile. The third and fifth rows show scores that reflect the user average ratings for different types of restaurants, normalized to~$[0,1]$. Note that properties are usually not recorded for every user, e.g., Carol has never rated Mexican food. The fourth and sixth rows show scores reflecting the relative frequency that each of the users visits different types of restaurants.
	%, where~1 means ``all the time'' and~0 means ``never''. 
\end{example}

In practice, user profiles can contain many properties. This may be due to a diverse activity of a user in the system (e.g., providing opinions about many types of destinations), %tracking data on different levels of granularity (e.g., Mexican food, South American food\dots) or 
due to various analyses performed over the data (e.g., one can compute the average rating, maximum rating\dots) and so on.TODO: change data In the dataset that we have constructed from the TripAdvisor website (see Section~\ref{sec:system}), each user has up to~2189 properties. In many scenarios, due to the inherent high dimensionality of user profiles, user profiles consists of merely a subset of available properties. This selection of specific properties and subsequent analyses has a major impact on the diverse set of procured opinions and therefore choices should be carefully made with respect to the desired diversification goals in the specific domain.
\section{diversity notion}
\label{sec:diversity}

The process of procuring diverse user opinions begins with capturing diverse user profiles. We conclude that in most cases the opinions of users regarding a specific topic or question is not available to us, therefore we suggest that a selection of diverse users profiles would yield diverse opinions. The concept of diversification has been widely researched (see Section~\ref{sec:related}), with many approaches suggesting the use of a \emph{similarity function} that quantifies the similarity between two objects. The notion of \emph{distance-based} diversity utilizes this measure to portray diversity as the problem of finding a set k objects for which the pair-wise similarity is minimal (i.e., the distance is maximal).

\begin{example}
	Following the example in \ref{tab:profs}, we could look at profiles as vectors of properties and naturally define a similarity function, e.g. euclidean distance. To measure the difference between non-scalar properties (e.g. location, age) we could use a constant value, 1 for example, to denote whether the values are different or 0 if they are identical. Now, the selected group of size 2 for which the pair-wise distance is maximal is \{Bob, Eve\} with a measured similarity of approx. 1.29.
\end{example}

Using distance-based techniques to acquire a diverse group of items is a very popular approach in many domains, but when it comes to gathering user opinions it would not yield the best results. For example, an entrepreneur looking to open a new restaurant does not necessarily care for the most eccentric and obscure opinions, and would rather get a picture that best resembles the range of opinions within the crowd.
When it comes to user opinions, we propose the concept of \emph{coverage-based} diversity which focuses on the representation of as many groups as possible within a source population. We later show that this approach does achieve superior results, both in qualitative and predictive tests. The reason coverage-based diversity produces objectively diverse opinions is due to the fact that different opinions inherently exist within the many groups inside the source population.
YAEL-TODO: add a segment about data ethics.
An ideal coverage of the entire population by a subset of users would be one where each of the different groups (e.g. users who visits Italian restaurants frequently, users who tends to dislike Mexican food) inside the population are proportionally represented. A similar approach has been adopted by surveyors in the form of a \emph{Stratified survey} [add-cite]. This form of surveys tries to reflect the inherent diversity of the source population by first dividing the population into non-overlapping groups and afterwards performing a random sample of participants from each group, with proportion to the group's size. In theory, this approach would produce the optimal coverage. In practice, it is generally impossible to obtain such a coverage due to these challenges:

\paragraph*{High Data Dimensionality} The different groups are based on user preferences, characteristics and activities which makes the user profiles. In reality, each profile could be made up from hundreds to thousands of properties, which brings the distinct total number of properties to acknowledge (and thus groups to cover) within the 100,000's. An attempt to \emph{proportionally} represent each and every group with only a selection of representatives is doomed to fail. There also exist a great difference of the groups sizes. For example, there are many more people who frequently visits Italian eateries than people who visits Indian restaurants. As a result, there is a long tail of tiny groups which is impossible to cover with only a few users. In our research, we propose a method which enables the ordering of groups according to priority, in an attempt to represent as many \emph{important} groups as possible.TODO: taxonomies.

\paragraph*{Range Coverage} Recall the model we presented in Section~\ref{sec:prelim}. Each property has a score within the range [0,1]. It is not feasible to try and represent each of the discrete scores in each of the properties. In the next section we demonstrate how to associate sub-ranges of [0,1] into different groups, in an attempt to accurately represent users with different opinions (which reflect in different scores).
\begin{example}
The property called "Average Rating of Mexican Food" has a score range of [0,1]. Suppose we divide the range into sub-ranges [0,0.5) with label "Low" and [0.5,1] with label "High". A user with a score of 0.7 will suffice as a representative for the group "High Average Rating of Mexican Food".
\end{example}

\paragraph*{Domain-Relevant Properties} 
In many cases, the importance of certain properties is tightly linked with a specific application. For instance, a French bistro restaurant owner seeking opinions regarding a new menu would give much more consideration to opinions of those who are fond of European cuisines. He could also decide to completely ignore people that never visited a French restaurant before. Notably, the diversity the user is aiming for is subject to the specific application rather than a general notion.
To capture domain related nuances we introduce a comprehensible \emph{explanations} and \emph{customizations} mechanisms. By allowing the user to better understand why a specific subset of opinions were chosen by our algorithm, we expect him to be competent to fine-tune the selection to a specific domain using designated customization tools.


% There are two key challenges here: (a) formally defining an adequate
% similarity measure, and (b) providing an efficient algorithm to
% compute it.


% Then, in the following section, we explain how
% to efficiently compute it.

%\paragraph*{Overview}
%Different similarity measures are considered in previous work
%(e.g.,~\cite{dinoia2012linked,hu2008matching,jean2007asmov,resnik1995using}),
%yet none of them can fully account for the semantically-rich representation of user data in our setting. First, \qlang{} requires a \emph{semantically aware comparison of \fset{}s}, e.g., quantifying the relevance to practicing yoga at a park (Figure~\ref{fig:qlang} lines~11-13) of a user like Adam who practices Pilates, another form of mind-body fitness, at Jardin des Tuileries (Table~\ref{tab:answers}) or of users who play chess and basketball outdoors, practice yoga indoors, etc. Second, our similarity measure must operate under the assumption that\emph{ profiles are incomplete}, e.g., it is possible that Adam does practice yoga in the park, but failed to mention it. Third, \emph{support scores} play an important role in similarity evaluation, and e.g., our measure should allow comparing Adam to Benjamin who does practice yoga at a park, but very rarely. %Our similarity metric should also enable the online evaluation of \qlang{} queries, which is discussed in the next section.


%., namely \fset{}s, possibly accompanied by support scores. Moreover, as explained in the Introduction, our measure is applied over user profiles that are allowed to be \emph{incomplete} and due to the interaction with the query language, may also be applied over \emph{profile parts}.
%For instance, reconsider the soft constraint asking that Isabella's
%match should practice Yoga on a park, in $Q_{\textrm{Isabella}}$. According to Adam's profile (Table~\ref{tab:answers}), he
%practices some other form of mind-body fitness, Pilates, at Jardin des Tuileries, with a certain frequency. To compute Adam's relevance to the query, one should be
%able to \emph{quantify} how similar this hobby of
%Adam are to Isabella's specified preference. This similarity measure should allow comparing Adam to Benjamin who does practice Yoga (at Jardin du Luxembourg), but very rarely, or to other possible users who play chess and basketball outdoors, practice Yoga indoors, etc.

%In comparison, Benjamin does
%practice Yoga on weekends, but according to the support score he
%does so very rarely, much less frequently than Adam practices
%Pilates. In a sense, Adam should ``gain points'' for frequently
%practicing and Benjamin should ``gain points'' for being closer to
%the desired description. How do users who play chess on weekends, or practice Yoga on weekdays, etc., compare to Adam and Benjamin?

%Our contribution in this section is thus to define \emph{formal
%constraints} on any similarity measure that takes into
%consideration the semantics of \fset{}s that describe user data as well as support scores. Then, by
%extending and adapting standard similarity functions to our data
%representation, we develop a \emph{generic formulation} that (i) satisfies the constraints, as
%we prove below, (ii) can be computed via an efficient algorithm, as
%we show in Section~\ref{sec:exec}, and (iii) works well in real-life
%scenarios, as we show in
%Section~\ref{sec:Implementation}. 

%Given two pieces of data, e.g., a soft constraint (looking  for a
%user who practices yoga on weekends) and a \fset{} with support
%(Adam practices Pilates on weekends with support~0.27), The semantic
%part of our definition requires identifying the information that is
%\emph{common to the two pieces}. In this example, the common
%information might be practicing a mind-body fitness on weekends, but
%may also be doing any activity on the weekends. We formally define
%common information using a notion of \emph{semantic subsumption}
%between facts and \fset{}s (see Section~\ref{sec:subsumption}).
%
%Next we quantify the meaningfulness of a piece of common
%information. E.g., for the purpose of evaluating similarity,
%practicing a mind-body fitness on weekends is more meaningful than
%reading a book, if we assume that significantly fewer people have
%the former habit. To measure this we use a notion of
%\emph{information content} (defined in Section~\ref{sec:ic}) based
%on~\cite{resnik1995using}.

%Our semantics- and support-aware similarity metric builds on two auxiliary notions: \emph{semantic subsumption}, capturing the s
%Our definitions use two auxiliary notions: \emph{semantic
%subsumption}, that allows identifying information common to two
%semantic units, and \emph{information content}, that quantifies how
%informative the common information is based on its prevalence. The
%first factor of our similarity measure, \emph{semantic similarity}, compares two semantic units
%based on their subsumption relationships and information content (Section~\ref{sec:icsim}).
%The second factor, \emph{support similarity}, completes the semantic similarity by
%considering support values (Section~\ref{sec:supsim}). Finally, we combine the semantic
%similarity and the support similarity to form a combined similarity
%measure (Section~\ref{sec:combined}).

%Reconsider query $Q_{\textrm{Isabella}}$ from Figure~\ref{fig:qlang}. In particular, $Q_{\textrm{Isabella}}$ selects users whose basic profiles resembles Isabella's, considering only the users' hobbies. Consider the sample basic profiles in Table~\ref{tab:profile} and observe that
%\emph{comparing these profiles textually is insufficient}: while Isabella likes Mathematician and Benjamin likes drawing, both of them (implicitly) like (some form of) visual art.
%To capture this similarity, we use \emph{semantic subsumption} between facts and \fset{}s (formally defined in Section~\ref{sec:subsumption}), to determine common information
%\emph{implied} by the two users.

%Having identified common information between (parts of) two profiles (or, a profile and a desired property), we need to \emph{quantify} how similar they are. For instance, Isabella and Benjamin's common interest in visual art is more meaningful for the sake of similarity than the fact they are respectively interested in Mathematician and basketball (which may only imply they both have \emph{some} hobby). To measure this we use a notion of \emph{information content} (defined in Section~\ref{sec:ic}) based on~\cite{resnik1995using}.
%
%
%Regarding the users' extended profiles, note that in Table~\ref{tab:answers}, both Adam and Benjamin practice (some form of) mind-body fitness on weekends. However, Adam's support value is higher (i.e., he practices more often), which should ``add points'' to Adam's
%similarity to Isabella's desired \fset{} (in lines ~11-13 in $Q_{\textrm{Isabella}}$) in comparison with Benjamin.
%
%
%\vspace{4px}
%The first factor of our similarity metric, \emph{semantic
%similarity}, described in Section~\ref{sec:icsim}, thus compares two
%semantic units based on
%their subsumption relationships and information content. The second factor,
%\emph{support similarity}, described in Section~\ref{sec:supsim},
%completes the semantic similarity by considering support values.
%Finally, we combine the semantic similarity and the support
%similarity to form a single similarity measure, in
%Section~\ref{sec:combined}.
% PREFIX ontology:  <http://xmlns.com/foaf/0.1/>
% PREFIX profiles:  <http://xmlns.com/foaf/0.1/>
% PREFIX histories:  <http://xmlns.com/foaf/0.1/>
% SELECT ?u
% WHERE {
%      ?x ontology:instanceOf ontology:Place .
%      ?x ontology:near       ontology:Paris .
%      ?u profiles:livesIn ?x.
% }




%observe comparing these profiles textually is insufficient: for instance, while Ann likes photography and Carol likes drawing, both of them (implicitly) have some form of art as a hobby. We would like this fact to ``add points'' to the similarity between Ann and Carol.
%
%
%To fully understand the semantics of a \qlang{} query, one must provide formulas for the similarity scores over \fset{}s and answer databases. On the one hand, we will use an independent module for computing similarity scores, which would enable one to plug in alternative modules that compute different similarity formulas. Similarity computation has been considered in previous work in various contexts (e.g.,~\cite{benticha2012user,dinoia2012linked,seco2004intrinsic}), and one may attempt to adapt them to our setting. However, on the other hand, no existing similarity measure can fully account for the complex semantics in our context in a generic manner (see Section~\ref{sec:related}). We therefor propose, in this section, a novel definition of similarity which considers the relations between semantic units in our setting, as well as support scores.

%We start by defining subsumption relations between semantic units, which will be used in defining the ``semantic'' part of the similarity sco
\section{coverage based diversity}
\label{sec:coverage} %In order to efficiently execute \qlang{} queries, one has to efficiently evaluate both hard and soft constraints (\textcd{WHERE} and \textcd{SIMILAR} clauses, respectively). Note
%thatWe are now ready to define our notion of diversity, as follows.

We define diversity based on user properties and their scores.  
Scores are crucial in this respect: e.g., it makes sense to group Mexican food lovers and dislikers separately, rather than grouping all the Mexican food reviewers together. For that, we split the range of scores of each property $p\in\mathcal{P}$ into a set of \emph{non-overlapping buckets} $\beta(p)$ (the bucketing method is described in Section~\ref{sec:compute}). A \emph{group} is then the subset of users with relevant property and score, formally,
\[g_{p,b}\colonequals \{u\in\mathcal{U}\mid D_u=\langle P_u, S_u \rangle \wedge p\in P_u \wedge  b\in\beta(p) \wedge S_u(p)\in b\}\]


\begin{definition}
	Let $\mathcal{G}$ be the set of all non-empty groups $g_{p,b}$ with respect to a given set of users $\mathcal{U}$ and profiles $\{D_u\}_{u\in\mathcal{U}}$.   
	We say that a subset of the users $U\subseteq\mathcal{U}$ \emph{covers} a group $g_{p,b}$ if $|U\cap g_{p,b}|>0$, and denote the set of covered groups by $\funct{cov}{U}$. Further let $W:\mathcal{G} \To \mathbb{R}^{+}$ be a \emph{weight function} indicating the importance of each bucket. Given also a budget $B\in \mathbb{N}$,  we define \texttt{MAX-DIVERSITY} as the problem of finding a subset $U\subseteq\mathcal{U}$ such that $|U|\leq B$ and the weight of~$U$, defined as $W_{\mathcal{G}}(U)\colonequals\Sigma_{g_{p,b} \in \funct{cov}{U}}W(g_{p,b})$, is maximized.       	
\end{definition}

The weight function $W$ captures the importance of different groups. A natural choice, which we employ in our implementation, is defining the weight of a group as the \emph{number of users} in it. The purpose of this choice is to increase the likelihood of covering large groups before small ones.   



\begin{example}
	Reconsider the user profiles in Table~\ref{tab:profs} and assume that each property is divided into two buckets: scores in $[0.5,1]$ (``high'') and scores in $[0,0.25]$ (``low''). The numbers in superscript show the weights -- i.e., number of users -- on the first occurrence of each property bucket. E.g., there is only one bucket with~3 users: \elem{avgRating Mexican high}. %Properties with~2 users include \elem{livesIn Tokyo high}, \elem{visitFreq Mexican high}/\elem{low} and so on. 
	In this case, the user group of size~2 that should be selected is \{Alice, Carol\} with sum of weights~18. %$2+1+3+2+2+2+1+1+2+2 = 18$. 
\end{example}

%We note that our metric is generally biased to select users with more properties, as they cover more buckets. However, this can be mitigated by properly selecting in specific context where selecting users with varied levels of activity is desired, by choosing to cover only relevant properties (level of activity, user rank and so on) which are available for all or most users.



%Our approach for property coverage considers each property individually and indepedndently. When properties are highly correlated in some population (e.g., users who frequently visit ``cheap eats'' and rarely visit ``fine dining'' restaurants), we assume that these properties are likely to occur together in selected users, but do not account for the combination explicitly. One could more generally consider the coverage of n-sized combinations of properties. However, in our experience, the contribution of such combinations to the selection was marginal (since most combinations are too rare, and thos who are frequent are likely to be encountered in any case) while the overhead of computing them was high. Another approach might involve the grouping of users via (high-dimentional) clustering techniques, and then choosing a representative for each group~\cite{??}. However, the results of such approach, particularly in high-dimensional data, are unintuitive to a human user.



%\noindent We use $\mathcal{G}$ to denote the set of all such groups. 

\paragraph*{Customization and Explanations}
The \emph{explanation} of a selected subset $U\subseteq\mathcal{U}$ is the partition of~$\mathcal{G}$ to $\langle \funct{cov}{U}, \mathcal{G}-\funct{cov}{U}$, namely, covered and non-covered groups in $\mathcal{U}$. The explanation of a selected user $u\in U$ is $\funct{cov}{\{u\}}$, namely, the groups covered by $u$'s properties. Intuitively, the properties of $\mathcal{P}$ are assumed to have meaningful names since they are used in user profiles, and score buckets can also be easily given names (``high'', ``medium''\dots), which yields meaningful names to  the groups of~$\mathcal{G}$. The client user can then see which groups are covered (and by which selected user) and which are not. (The UI of \sysname{} further enables easy and intuitive browsing through these groups, see Section~\ref{sec:system}.)

A \emph{customization feedback} of the user is composed of four distinct subsets of $\mathcal{G}$, denoted $\mathcal{G}_{+},\mathcal{G}_{\textrm{--}},\mathcal{G}_{d}$ and~$\mathcal{G}_{d?}$. $\mathcal{U}$~is refined to consider only users of interest $\mathcal{U}_{\name{c}}$, who belong to every group in  $\mathcal{G}_{+}$ (if $\mathcal{G}_{+}$ contains more than one bucket of some property $p$, users need only belong to one of them) and to none in $\mathcal{G}_{\textrm{--}}$. Formally,
\[\begin{split}
\mathcal{U}_{\name{c}} = & \{u\in\mathcal{U}\mid\forall g_{p,b}\in \mathcal{G}_{+}, ~\exists b'\in \beta(p): ~u\in g_{p,b'} \wedge g_{p,b'} \in\mathcal{G}_{+}\} \\
& \cap  \{u\in\mathcal{U}\mid\forall g_{p,b}\in \mathcal{G}_{\textrm{--}} : ~u\not\in g_{p,b}\}
\end{split}\]

The customized diversity problem  \texttt{MAX-DIVERSITY-C} is then to select new subset $U\subseteq\mathcal{U}_{\name{c}}$ that maximizes $W_{\mathcal{G}_{d}}(U)$, namely, the sum of weights over covered groups from  $\mathcal{G}_{d}$, breaking ties by $W_{\mathcal{G}_{d?}}(U)$. Note that \texttt{MAX-DIVERSITY-C} can be easily reduced to \texttt{MAX-DIVERSITY} by a proper selection of weight function.

\begin{example}
	Reconsider the problem of selecting a user subset of size~2 from Table~\ref{tab:profs}, but now assume the client prefers users from diverse locations and people familiar with Mexican food. This translates to a feedback where $\mathcal{G}_{+}$ consists of the two buckets \elem{high} and \elem{low} of \elem{AvgRating Mexican} (requiring the users to have any rating for Mexican food), and $\mathcal{G}_{d}$ consists of the different \elem{livesIn~<city>} properties. $\mathcal{G}_{\textrm{--}}$ and $\mathcal{G}_{d?}$ would be $\emptyset$ and $\mathcal{G}-\mathcal{G}_d$, respectively. Then, the refined user set $\mathcal{U}_{\name{c}}$ will exclude Carol who did not rate Mexican food. The best user subsets will now be \{Alice, Bob\} or \{Bob, Eve\}, which maximize the sum of weights over \elem{livesIn~<city>} properties (to~3) and among other subsets that achieve this maximum (e.g., \{Alice, David\})  the former two subsets further maximize the sum of weights over other properties (to~14).
\end{example}


\subsection{Diversity Computation}
\label{sec:compute}

We next consider the computation of a diverse subset of users. 

\paragraph*{Computing $\mathcal{G}$.} The set of properties $\mathcal{P}$ is assumed to be given (derived from the user profiles). To compute the buckets $\beta(p)$ for any property $p\in\mathcal{P}$ we first determine the number of buckets heuristically as $\lceil\log\name{user}(p)\rceil$, where $\name{user}(p)$ is the number of users with property $p$ in their profiles. For sanity, we bound this number from above by $\name{uniq}(p)$, the number of unique scores obtained for~$p$. We then use one-dimensional clustering (based on k-means) to split the score range $[0:1]$ into buckets.

% \(\max\{\min\{\name{uniq}(p), \lceil\log_{10}\name{user}(p)\rceil\},2\}\), where $\name{uniq}(p)$ is the number of unique score values obtained for $p$ and $\name{user}(p)$ is the number of users with property $p$ in their profiles. For Boolean properties we always use two buckets for~1 and~0. For other properties we use one-dimensional clustering (k-means) to split the range $[0:1]$ to buckets.

\paragraph*{Solving MAX-DIVERSITY} Unsurprisingly, achieving an optimal solution is intractable unless P=NP, even for simple weight functions. The decision problem \texttt{DEC-MAX-DIVERSITY} corresponding to \texttt{MAX-DIVERSITY} is that of the existence of a subset of a given cardinality $B$ such that the sum of weights of covered groups exceeds a threshold $T$. We can then show:   

\begin{proposition}
	DEC-MAX-DIVERSITY is NP-complete in $B$.  
\end{proposition}

Despite this intractability result, our problem enables efficient approximation, due to properties of $W_{\mathcal{G}}(U)$. First, $W_{\mathcal{G}}(U)$ is \emph{submodular}, namely, for any $U\subseteq U'\subseteq \mathcal{U}$ and $u\in\mathcal{U}$ it holds that $W_{\mathcal{G}}(U\cup\{u\})-W_{\mathcal{G}}(U)\geq W_{\mathcal{G}}(U'\cup\{u\})-W_{\mathcal{G}}(U')$. Since the group weight function $W$ is non-negative, we get that $W_{\mathcal{G}}(U)$ is also non-negative ($W_{\mathcal{G}}(U)\geq 0$) and \emph{monotonous} (if $U\subseteq U'$ then $W_{\mathcal{G}}(U)\leq W_{\mathcal{G}}(U')$). The size of the groups that we consider is bounded by $B$. For such functions, a greedy algorithm that iteratively adds one user $u$ to the selected subset $U$ so as to maximize $W_{\mathcal{G}}(U\cup\{u\})$ is known to guarantee a good approximation ratio ($1-1/e$)~\cite{nemhauser1978analysis}. We implement such a greedy algorithm in \sysname{}. 



%A function over a subset $U\subseteq\mathcal{U}$ is said to be \emph{submodular} if  an \emph{approximate} solution may be efficiently computed under plausible assumptions on the weight function. We say that a function is sub-modular if \scream{complete..}

%\scream{Give the algorithm for sub-modular + guarantees if any + proposition that the weight function you use is sub-modular.}  

%\scream{Discuss Bucketing}





%
%We say that a subset of the users $U\subseteq\mathcal{U}$ \emph{covers} a group $g_{p,b}$ if $|U\cap g_{p,b}|>0$. Ideally, one would like to cover all the groups by selecting up to some fixed bound $\top$ of users from $\mathcal{U}$. Unfortunately, the corresponding decision problem -- Can a set of groups be covered by $\top$ users? -- is equivalent to Set-Cover, and is thus NP-hard in $\top$. 
%
%Even if $\top$ is small enough to enable the enumeration of all $\top$ subsets of $\mathcal{U}$, it may still be impossible to cover all the groups since the data is high-dimensional ($|\mathcal{P}|>> \top$).
%We therefore relax the requirement and aim at obtaining the ``best'' possible cover. For that, we also define a \emph{weight function} and consider the problem of selecting a group of $\top$ users that \emph{maximizes the aggregated weight of covered properties}. %Since some properties (and buckets thereof) are more frequent than others, we use the number of users in the bucket as a weight. 
%In the current implementation of \sysname{} we choose a simple weight function that captures the \emph{number of users} in each group and aim to maximize the sum of weights of covered groups. The purpose of this choice is to increase the likelihood of covering large groups before small ones. %Note that this function ignores possible overlaps between groups. While this could be easily 




\section{Experimental Study}
\label{sec:Implementation} We developed a prototype engine for
parsing and evaluating \qlang{} queries. This engine uses preprocessing of the term taxonomy to speed up LCA computation, which is at the core of similarity computation (Section \ref{sec:exec}). A further speed-up is achieved by distributing, over multiple cores, the computation of similarity between different user pairs, which is virtually independent. Finally, the engine employs a dedicated caching mechanism to compactly store intermediate results and avoid unnecessary computations, based on Observation~\ref{obs:cash}.
% and implemented a caching mechanism that stores previous calculation results and avoids calculating unnecessary ones, . 
The prototype is implemented in Java, using
the Apache Jena library \cite{jena} for 
%handling RDF data and for
 a SPARQL engine. %and the JGraphT library (\url{http://jgrapht.org}) for graph processing operations on the ontologies.
%Using this engine, we conducted an experimental study designed to assess the flexibility, accuracy and efficiency of our approach for real user data.


\paragraph*{Goals}
%We set three goals for the experimental study. 
%Since user selection using a semantically-rich setting like ours has not been studied before, 
Since general-purpose user selection framework has not been studied before, no standard benchmark over which one could test the full capabilities of \qlang{} was available. We have thus constructed two benchmark datasets using real-world data from the following sources: Stack
Overflow (SO)~\cite{StackOverflow}, a large Q\&A platform
for computer programming, and AMiner~\cite{tang2008arnetminer}, an
academic social network containing author profiles along with collaboration
and publication data. These datasets provide natural scenarios for user selection, which we have captured via \qlang{} queries.
%\footnote{In fact, AMiner uses data from DBLP, which is often used in link prediction~\cite{Jeh:2002:SMS:775047.775126}, one of our experimental scenarios.} 
%as detailed below.


%We examined our framework as a whole, as well as the contribution of eachof its components. 
%We focused on several typical scenarios for user
%selection where quality could
%be evaluated with respect to some ground truth. 
%We compared the results of \qlang{} to
%common alternative strategies from machine learning (SVM) and graph
%processing (SimRank~\cite{Jeh:2002:SMS:775047.775126}).

Our first goal was to \emph{evaluate the result quality} that our solution achieves, i.e., the adequacy of the selected users. We compared the results to common alternative strategies from machine learning, graph processing, and collaborative filtering. To examine
the contribution of components within our solution,
%, and in particular the ingredients of our dedicated similarity measure, 
we
implemented restricted variants of \qlang{} that do not include some
components, or use alternative similarity measures. 
Finally, we conducted a user study aiming to assess the relevance of results by real users.
Our second goal was to \emph{test the system's
running times and its scalability}. We examined the execution time of
\qlang{} queries for
varying input sizes.
%As a complementary case study,
Lastly, we examined the \emph{flexibility} of \qlang{} and its ability to express specific user selection needs in typical contexts, including recommendations, predictions and expert finding.

%Our experimental results indicate that \qlang{} is expressive and scalable
%enough to be practically used for selecting users in various
%contexts. While our solution is declarative and generic, it  achieves high quality results, even compared to task-dedicated,
%non-declarative alternatives. This is true even though we have manually and intuitively written the queries (as a real user would have done) rather than
%carefully tailoring each query to outperform the competitors.




% Note that for the following scenarios in \ref{experiments}, the \qlang{} queries were chosen intuitively and heuristically with respect to their intent, exemplifying the declarative and straightforward nature of our language. While we believe that improving the queries will yields better prediction results, we leave the \qlang{} optimization for future work.


\subsection{Experimental Setup}
\label{experiments} We describe how the data was extracted and
converted into our representation, then present the different baselines tested. To allow using standard RDF tools, the system generates additional RDF data that explicitly captures information about facts and \fset{}s, e.g., \{\elem{FactSet456} \elem{hasSupport} \elem{0.01}\}. The system then converts the selection clauses in a given \qlang{} query (\textcd{From}\dots\textcd{WHERE} and \textcd{RESTRICTED TO} clauses) into SPARQL queries over the resulting RDF repository.
All experiments were conducted on a Linux server with~$24$ cores,
a~$2.1$GHz CPU and~$96$GB memory. 



%\vspace{-4px}
\paragraph*{Stack Overflow dataset}
In this popular Q\&A platform, user questions are associated with a set of tags that reflect their topics.
%each question is posed by a specific
%user, and associated with a set of tags that reflects its topic.
% and assists users in searching relevant questions for them.
Among its answers, a question may have one designated answer chosen
as the most accurate one. Each user has a 
profile with properties such as name, registration date, reputation score, etc. We collected over~$900$K questions in~$300$ popular topics (tags), asked
by over~$175$K users. We then gathered their
personal profiles, and collected more than~$2.3$M previous answers
of those users, to assemble a coherent subset. The retrieved data
was naturally mapped into our model: we constructed the basic user profiles from the extracted profiles. 
The extended profile
of a user was constructed from her tags to which she
contributed, with a support value that reflects the portion of her
contribution to that tag, among all tags. The ontology was constructed from DBpedia~\cite{dbpedia}. We aligned each
tag to its matching concept and used the relevant part of DBpedia taxonomy.  
The resulting database consists of $20$M entities.



%\vspace{-6px}
\paragraph*{AMiner dataset}
We extracted the data of~$1.7$M computer scientists 
and considered their publications between~$2005 - 2015$ (over~$2$M papers). The basic profiles of the authors consist of their affiliation, h-index, fields of interest, and some publication data (e.g., title and year). The extended profile of each author records her
co-authorships (with a support value reflecting the fraction of her publications with each co-author), and publication venues (with support value reflecting the fraction of her publications in each venue). The domain ontology was built as
described above, aligning AMiner terms (affiliations, fields of
interest, etc.) with DBpedia concepts, and constructing the
corresponding taxonomy. The resulting database consists of $16$M entities.
%$35295$.



%\vspace{-6px}
\paragraph*{Alternative algorithms} \label{setup}
We have compared the results of \qlang{} to several competing baselines, as detailed in Table~\ref{table:baselines}. The first three baselines (\textbf{SPARQL}, \textbf{No Support}, \textbf{No \Fset{}s}) are restricted variants that serve to examine the contribution of our framework's components compared to the framework as a whole.
%As mentioned above, our goal was
%to examine the contribution of each of the individual components of \qlang{} as well as the performance of our framework as a whole. 
%\textbf{1.SPARQL}. User selection is done by a standard SPARQL
%query, that is, \emph{only hard constraints are employed}. This allows for evaluating the need of similarity-based soft
%constraints in user selection. 
%\textbf{2.No Support}. User selection is done by the same \qlang{} queries, but
%the support values are ignored. This allows for evaluating the need of considering support values in user selection,
%and the effect of our support-similarity measure. \textbf{3.No \Fset{}s}. User selection is done by the same \qlang{} queries, but
%facts are considered individually and co-occurrence of facts is
%ignored. This allows for evaluating the need for our rich data model
%and its corresponding semantic-similarity measure. \textbf{4.SimRank}. We consider SimRank \cite{Jeh:2002:SMS:775047.775126}, a
%similarity measure which is commonly used in social networks for
%link prediction problems. We used an highly efficient approximation technique suggested in \cite{tian2016sling}. \textbf{5.SVM}. Users are selected using SVM, a common machine learning method for such a task.
% We experimented with
%regression and classification-based variants and describe here the best performing 
%classification variant.
%We used the scikit-learn implementation \cite{SVM}.
%For each scenario we employed an
%SVM classifier, where each user represents a class, and the algorithm
%computes the probabilities of a class to be chosen.  
%This
%baseline is used to evaluate the ability of \qlang{} to capture
%common user selection scenarios, which can alternatively be specified
%as machine learning problems.
The \textbf{SimRank} and \textbf{SVM} baselines are used as alternative similarity measures, i.e., hard constraints are still used for initial data filtering. To implement the SimRank measure, we used an efficient approximation technique suggested in~\cite{tian2016sling}, and used the scikit-learn implementation~\cite{SVM} for the SVM. 
%The results of these tools without initial filtering were inferior in all the experiments.
Additional similarity measures were examined, including the cosine and Jaccard
measures used in collaborative filtering. Due to their consistently inferior results, they are omitted from presentation.



\vspace{-1mm}
\subsection{Qualitative Experiments}
\label{sec:experiments}
%We next present our experiments over the two datasets, along with an
%analysis of the results and performance. 
%To test the flexibility of \qlang{}, we collected various scenarios that are considered in user selection literature, and composed \qlang{} queries that capture these scenarios in our datasets. Notably, the use of soft constraints in all
%scenarios was natural and often necessary. %For space constraints we defer details to~\cite{fullpaper}. 
While \qlang{} is able to express a wide range of user selection scenarios (see Section~\ref{sec:flexibility}), to examine the adequacy of selected users we focused on specific cases for which there exists a \emph{ground truth} that enables assessing the results quality. For each scenario, we describe the used queries and compare our results with the alternatives and ground truth. %We then examine the scalability of our system.

\begin{figure}
{\scriptsize
\begin{Verbatim}
SELECT ?u
FROM basic-profile(?u) WHERE
    \{?u creationDate ?d. Filter (?d < 19.6.2015) \}
SIMILAR basic-profile(?u) TO basic-profile(Basil_Bourque)
    WITH SIMILARITY AS profSim > 0
SIMILAR extended-profile(?u) TO 
                 \{?u answeredOn Java. ?u answeredOn I/O \}
    WITH SIMILARITY AS topicSim > 0.2
ORDER BY AVG(profSim, topicSim) LIMIT 30
\end{Verbatim}
} \vspace{-6mm} 
\caption{User selection for Stack Overflow
question.} \label{fig:SOquery}
%\vspace{-4mm}
\end{figure}


\paragraph*{Stack Overflow experiment}
\label{SO} The setup we have focused on for the SO experiment is as follows (see details below).
\begin{compactitem}
\item \textbf{Task:} find users to answer a given question.
\item \textbf{Our query:} select experts for the question topics ($=$high support) who are also similar to the asker.
\item \textbf{Adjustments:} use only data generated before the question posting time.
\item \textbf{Evaluation:} \% of~500 random questions where designated answerer (ground truth) appears in top-$k$ selected users.
\end{compactitem}
We tested several alternative \qlang{} queries including the retrieval of top-ranked users for each of the
question's tags, i.e., retrieving the experts in a given topic. The eventually chosen query form requires, beyond relevant expertise (which was specified as a soft constraint), that the user asking the question and the one
answering have similar interests, which presumably increases the likelihood that the latter will be willing to answer the former's question. %It seems that a ``peer'' is more likely to provide a good answer than someone with significantly higher or lower capabilities. 
We constructed a template \qlang{} query that captures this user selection approach
and instantiated it for different pairs of questions and
questioners. 

\begin{figure}
	{\scriptsize
		\begin{Verbatim}
SELECT ?u
FROM extended-profile(?u) WHERE
	\{?u collaboratedWith ?v. \}
FROM extended-profile(?v) WHERE
	\{?v collaboratedWith Tova_Milo. \}
SIMILAR basic-profile(?u) TO basic-profile(Tova_Milo)
	WITH SIMILARITY AS profSim > 0
SIMILAR extended-profile(?u) TO extended-profile(Tova_Milo)
	WITH SIMILARITY AS topicSim > 0
ORDER BY AVG(profSim, topicSim) LIMIT 30
		\end{Verbatim}
	} \vspace{-6mm} \caption{User selection for AMiner
	question, instantiated for the user Tova Milo.} \label{fig:AminerQuery}
%\vspace{-4mm}
\end{figure}

For example, the question
posted on $19.6.2015$ by the user Basil Bourque: ``How do I read from a file in Java?'', tagged
by 'Java' and 'I/O', yielded the \qlang{} query in Figure~\ref{fig:SOquery}.




%\begin{example}
%\label{example}
%For the question
%posted on $19.6.2015$ by the user Basil Bourque: ``How do I read from a file in Java?'', tagged
%by 'Java' and 'I/O', the corresponding \qlang{} query choosing users to answer it, is depicted in Figure
%\ref{fig:SOquery}.
%\end{example}
%We randomly selected $500$ questions posed in $2015$-$2016$
%and examined the corresponding \qlang{} queries, instantiated from our template. %for choosing adequate responders to each question.
%Each of these queries was executed over data collected \textit{before} the question was posted. 
Recall that a question in SO has one designated
answer, chosen as its most accurate answer. To evaluate the results quality, we examined how the designated answerer was ranked according to our query and the alternatives, and counted the
percentage out of~500 queries where this user was among the top $k$ users selected, for varying $k$ values. For the SVM baseline, we modelled
each user as a class and mapped  questions into those classes using the question's and asker's
features. In the SPARQL variant, 
users were ranked by their reputation score.



Our main findings can be summarized as follows.
\begin{compactitem}
\item \qlang{} outperforms all the alternatives; closest is the (specially-trained) SVM.
\item Selecting the ground truth user is generally difficult; many adequate users may not see a given question.
\item Addressing ``cold start'' issues by completing missing information can improve the results.
\end{compactitem}
Figure~\ref{subfig:so_predictions} shows the results for two representative $k$ values that capture the general trend: in all cases, our solution outperforms the
competitors, and for larger $k$ values, the results of all algorithms improve, overall maintaining the observed differences between them. In particular, for $42\%$ of the
examined questions the \qlang{} query ranked the designated
answerer among the top $30$ recommended users, and for $51\%$ of the
questions among the top $100$ users.
This is
extremely positive given the task difficulty: our ground truth, the designated answerer, may be just one out of many users that can potentially answer the question.
This indicates that \qlang{} can
be used to add a more active ``answer request'' module to  SO or similar platforms.

Furthermore, the three restricted
variants of \qlang{} (SPARQL, No Support, No \Fset{}s) achieve
inferior results, demonstrating respectively the importance of soft
constraints and of analyzing support scores and \fset{}s. 
SimRank also shows significantly inferior results, since unlike our similarity measure, it does not exploit semantic information. SVM is
closer to \qlang{} in terms of quality, but unlike the customizable \qlang{}, this classifier
was specifically trained for this scenario and cannot be trivially adapted to other scenarios.



The cases
where the designated answerers were ranked low or not
selected at all by our query typically occurred due to missing
data: in over $70\%$ of the cases, either the asker or the designated answerer were new users with
little or no past activities. To further improve the results, an application owner may address such ``cold start'' problems by actively asking new
users for missing information, or by using information from
external sources.  %We leave this for future research.




%\vspace{-2px}
\paragraph*{AMiner experiment}
\label{AMiner} For this dataset, we focus on the following typical scenario (see details below). 
\begin{compactitem}
\item \textbf{Task:} identify potential collaborators for a researcher (a typical link prediction problem in social networks).
\item \textbf{Our query:} collaborators of collaborators for this researcher that also resemble her/his profile.
\item \textbf{Adjustments:} use data up to~2013 for querying, and later collaborations as the ground truth to be predicted.
\item \textbf{Evaluation:} (i) for~150 random researchers, the precision and recall of top-30 potential collaborators with respect to ground truth; and\newline (ii) for~27 researchers, the precision of top-10 potential collaborators in a real user study (\qlang{} only).
\end{compactitem}
%that involves identifying potential collaborations for a
%given researcher --  a typical \textit{link prediction} problem in social networks. % (See~\cite{fullpaper} for other examples.)
%Our query selects, for a given author, the top-$30$ users in her
%``friends of friends'' community (i.e., among those users that had
%previously collaborated with at least one of her co-authors) with the
%most similar profiles. 
See Figure \ref{fig:AminerQuery} for example query, using prof. Tova Milo as the target user.  %Similarity here is measured w.r.t. both the
%basic profile (i.e., fields of interest, h-index, etc.) and the extended profile (i.e., past
%collaborations and conference contributions).
In this experiment, beyond the evaluation against actual collaboration data, we conduct a user study to evaluate the result quality as perceived by clients.
%To examine the adequacy of the selected users we conducted
%two experiments: one comparing the results to actual
%collaboration data, and the other via a user study.

For experiment (i) we split the data into two parts: one
contains the publications up to~2013 (to be queried), and the second contains the
publications in later years (ground truth). We focused on authors with
$\geq10$ publications/ collaborations %and whose profile contains at least %one field of interest 
to avoid cold start issues. 
%We have randomly selected~$150$ such authors, and for each author we executed the corresponding query on the $\leq2013$ data and compared the results with actual collaborations in the following
%years. 
The SPARQL baseline here orders authors by their h-index, and
for the SVM classifier, each user represents a class and is modelled by a vector
of her fields of interest, past collaborations and
publications. 



Our main findings can be summarized as follows.
\begin{compactitem}
\item \qlang{} outperforms all the alternatives in both precision and recall; closest is SimRank, a common measure in collaboration networks.
\item In our user study, a small set of~10 selected users was sufficient to identify relevant recommendations for the vast majority of participants.
\item Completing missing information and correcting unclean data may serve for result improvement.
\end{compactitem}
The precision and recall of query results w.r.t. the ground truth,
%(namely, how many of our predictions were fulfilled and how
%many of the true collaborations were predicted, resp.)
averaged over~150 random authors, are plotted in
Figure~\ref{subfig:aminer_predictions}.
For over~$90\%$ of the examined authors, the \qlang{} query has identified at least one true future collaborator, and for over~$85\%$ of them $\geq 3$ of the predictions were correct.
Furthermore, for over $60\%$ of the authors, $\geq 50\%$ of
their true future collaborators were predicted by \qlang{}.
%, i.e.,
%the recall was $\geq 0.5$. 
%In contrast, the restricted variants of
%\qlang{} achieved at most~$0.4$ precision and at most~$0.52$ recall.
%In this experiment SimRank competes better than the former one, since it is designed for social network analysis (in this case, the collaboration network). Interestingly, the generic \qlang{} still outperforms the techniques that is specifically tailored to the given context.
%Here again, our solution outperformed all of the alternatives, and in particular SimRank, a common used measure in the context of collaboration networks. 


In our user study, we recruited $27$ known researchers (averaging h-index of $35$). For each researcher, we executed our \qlang{} query to select~$10$ possible collaborators, excluding past co-authors (assuming researchers would not object to collaborate with past co-authors), and asked the participants to estimate
how many of the selected users may be potential future collaborators
(precision). Figure~\ref{subfig:aminer_recommendation} depicts the results. Note that in this study, we could not have estimated the recall, since it would require researchers to list all of their potential future co-authors, which is naturally
unknown at a given moment. In a real-life usage of the system, users may further personalize
the query, adding individual filtering conditions on
location, language, etc. Yet, even our simple, generic query obtained
some interesting results. Indeed, $89\%$ of the participants found
at least one recommendation relevant, 
and $85\%$ of the participants found at least $3$ recommendations
out of $10$ relevant.

We further analyzed the false positives for this study. In
particular, only in $3$ out of the $27$ cases, the authors found the
recommendation irrelevant. These cases appear to be a consequence of
incomplete data (e.g., missing facts in both the basic and extended profiles) or unclean data (e.g., authors with basic
profiles that do not reflect their fields of interest). As mentioned
previously, such problems may be addressed by enriching the data using tools for building profiles from social networks~\cite{Difallah:2013:PTM:2488388.2488421}.


\vspace{-1mm}
\subsection{Scalability Evaluation}
%We next measure the runtime of our system, focusing on our new modules since we use an existing state-of-the-art engine to execute hard constraints. We consider the effect of three factors on the execution time: 
%(1)~the number of users admitting the hard constraints, on which our new modules operate in practice; (2) the size of the examined profiles and (3)~the shape of the ontology (i.e., its width and depth). We noted that in the quality assessment queries (previously described), the average number of users admitted the hard constraint was $1$K and the average number of triples in each user profile in our datasets is $100$ triples. The average execution time of these queries took less then $3$ seconds. We next used these queries to examine the effect of different parameters on the execution time, while relaxing some hard constraints to enable larger number of users to pass the hard constraint phase. 

Our quality assessment queries (previously described), had an average execution time of less than~$3$ seconds. 
We have further synthetically modified different parameters of our setting, by modifying the used queries, to observe their effect on performance and scalability:
%This was done these queries by relaxing the hard constraints to control the number of users admitting them, and examine the effect of different parameters on the execution times: %, by e.g. relaxing the hard constraints to control the number of users admitting them. 
%We considered the following effect on execution time: 
(1)~the number of users in the database, and in particular the percentage of users admitting the hard constraints; (2)~the size of the examined profiles and (3)~the
shape of the ontology in terms of width and depth. For comparison, in the previous experiments, the average percentage of users that admitted the hard constraints was~$\sim$1\% in the SO experiment and lower in AMiner, each profile contained~$\sim$100 RDF triples on average, and the average width and depth of the ontology part used in each query was~$\sim$50 and ~$\sim$7, resp. In the following experiments, when varying the value of some parameter, we fix the others using these average values.

 \begin{table}
 	\hspace{-1mm}
 	{\scriptsize
 		\begin{tabularx}{1.0\columnwidth}{p{0.25\columnwidth}p{0.38\columnwidth}p{0.26\columnwidth}}
 			\toprule
 			\textbf{Use Case} & \textbf{Example} & \textbf{Required \qlang{}\newline features} \\
 			\midrule
 			\textsf{1. Expert finding} & Select users highly linked to the key term ``Database" and who frequently publish in a top DB conference. See Figure~\ref{fig:experts}. & Combined similarity, order by selected values.\\
 			\midrule
 			\textsf{2. Link prediction} & Select co-authors of co-authors of a researcher, ranked by their profile similarity, as her potential future collaborators. See Figure~\ref{fig:AminerQuery}. & Hard constraints, combined similarity, order by similarity.\\
 			\midrule
 			\textsf{3. Profile-based\newline recommendation} & Select users with similar (extended) profile. See Figure~\ref{fig:qlang}. & Combined similarity,  restricted parts of profiles. \\
 			\midrule
 			\textsf{4. Context-based\newline recommendation} & Select users that are highly relevant to the tags ``Java'' and ``I/O'' based on their answers since 2015. See Figure~\ref{fig:SOquery}. & Hard constraints, combined similarity, order by similarity. \\
 			\midrule
 			\textsf{5. Community\newline Extension} & Select users who frequently interact with as many members of a given community. See Figure~\ref{fig:community}. & Combined similarity, order by similarity.\\
 			\bottomrule
 		\end{tabularx}
 	}
 	%     \centering
 	%    \includegraphics[width=1\linewidth]{figures/expresivness.pdf}
 %	\vspace{-3mm}
 	\caption{Use cases of \qlang{}.}
 	\label{fig:use cases}
 %\vspace{-2mm}
 \end{table}

We do not report here the running times of the SimRank and SVM baselines, since they were significantly inferior. Unlike \qlang{}, which is dynamically evaluated over the queried profile data, these alternatives require a preprocessing phase per query (sampling random walks for SimRank\footnote{Even the state-of-the-art optimizations techniques suggested for SimRank require a pre-processing phase ~\cite{tian2016sling, jiang2017reads}.}, and training a model for the SVM classifier). Re-executing this phase alone required above one minute per query, and overall could not compare with the scalability of \qlang{}.

Our main findings in this evaluation are as follows.
\begin{compactitem}
\item The execution time of \qlang{} grows sublinearly with the size of database/user profiles, due to our caching mechanism (which also achieves a $\geq\times5$ speedup).
\item Overall, execution takes a few seconds or less for every examined setting, allowing online query evaluation.
\item In terms of the input taxonomy, our polynomial bound from Prop.~\ref{prop:comp_all} is not met in practice, but rather our preprocessing of taxonomy makes execution times oblivious to its shape and size.
\end{compactitem}
Figures~\ref{fig:running times}(a) and~\ref{fig:running times}(b) depict the average running time for SO and AMiner queries, w.r.t.\ the database size, and for different ratios of users admitting the hard constraints.  
%In all cases, the average execution time took approximately $15$ seconds. However, in a real life scenario (as the dating service example described in the introduction), a reasonable number of potential users would be no more than couple of thousands ($5\%$ of $1.7$M is $85$K).
%In all cases, the average execution time, over all queries and percentage of examine profiles, took $\sim$15 seconds. However, note that in many real-life scenarios (e.g.\ in collaborator prediction) no more than several thousands users admit the hard constraints, in which case the execution times of our system can reach 1-3 seconds.
The sublinear growth (roughly square root) in execution times demonstrates the effectiveness of our caching mechanism (Section~\ref{sec:exec}). Since this mechanism leverages intermediate results to save computations, its effect is increased when the number of users, and hence the overlap between their profiles, increases. In comparison, the execution time we observed without caching (omitted from the graphs) increases linearly with the number of users, and was at least~5 times slower for any of the settings we have tested.

%Interestingly, one can see the effectiveness of
%our caching mechanism and the very moderate linear increase
%in execution time that it entails. The achieved speedup, that was consistent across all our experiments, was approximately $5$ times faster when using caching (for lack of space, we do not present here the graphs). That is, our dedicated caching mechanism significantly affects both the absolute running times and the scalability of our computations.



  %a reasonable number of potential users would be no more than couple of thousands ($5\%$ of $1.7$M is $85$K). 

%To further analyze the parameters affecting the execution time
Next, to examine the effect of the profile size on the execution times, we altered the hard constraints of the queries to include each time a fixed profile size between~$10$ and~$1000$ RDF triples. Figure~\ref{fig:running times}(c) depicts the effect of the profile size on execution time, again exhibiting a sublinear dependency achieved by our caching mechanism.%Observe that in cases where profiles consist of~$\sim$100 triples, (as common in real world scenarios) the average execution time is $\leq 3$ seconds. 
%Observe that the profile sizes has a linear effect on execution times. However, in real-life scenarios (as in AMiner and SO datasets), this parameter does not goes beyond several hundreds, i.e., the profiles typically consist of no more than few hundreds triples. 
%\scream{What is the bottom line? what do we want to say here? The execution time grows sublinearly with the profile size, which is good but why? our lca algorithm? caching?} 

Last, we have examined the effect of the ontology shape (width and depth) on query execution times, and showed that there is no significant effect (see Figure~\ref{fig:running times} (d) for width; similar results were observed for height). This is consistent with our theoretical analysis.


%To conclude, these experiments demonstrate that our solution is highly scalable and provides fast execution times.
%Overall, the experiments show that the our framework is indeed efficient in evaluating user selection queries, and in particular, allows online computation of our similarity measure. %Our dedicated caching mechanism makes the solution highly scalable.

\subsection{Flexibility: A Case Study}
\label{sec:flexibility}


We next present prominent user selection scenarios, exemplify them using queries over the SO and AMiner datasets, and analyze the use of the \qlang{} constructs in each scenario. Table \ref{fig:use cases} lists several common user selection scenarios. Each scenario is provided with a concrete example, demonstrating a typical ``information need'', and
the features required for its formulation in \qlang{}. In particular, note that all example scenarios require the use of (combined) similarity, indicating the importance of soft constraints.  

%We now look into the flexibility of \qlang{}. 
%We collected prominent user selection scenarios in real-life applications, exemplified them using queries over the SO and AMiner datasets, and analyzed the use of the \qlang{} constructs in each scenario. 
 


%In particular, all the example queries use soft constraints and hence implicitly require our semantic similarity measure.

\textsf{1. Expert finding.} As described in Section~\ref{sec:related}, much effort is made to determine a user's level of expertise and thereby identify the experts in some topic. 
%First, note that the definition of an `expert' may vary across applications. For instance, SO ranks experts in a topic by the scores that other users gave to their answers related to that tag (i.e., reputation score)
%Our goal is not to devise a new definition but rather enable expressing a wide range of such definitions via \qlang{}. For instance, SO ranks experts in a topic by the scores that other users gave to their answers related to that tag (i.e., reputation score). 
%Obtaining the same ranking in \qlang{} can be done via a simple query that selects the scores of users in a give tag and orders them accordingly. 
%When such scores are not available, experts can be identified by their \emph{similarity to selected properties} that are related to the topic in question. 
For example, consider the selection of database experts from the AMiner dataset. %An expert database researcher may be defined as a person who (a) specializes in databases (or related topics), (b) frequently publishes at a top-tier conference in the field and (c) has a high h-index. 
Figure~\ref{fig:experts} depicts a \qlang{} query attempting to address this task by selecting researchers who specialize in a subject similar to ``Databases'' and have published in some top DB conference, ordered by their h-index.
To verify that this query indeed retrieves database experts, we presented the top-$10$ results to~$10$ different database researchers. For comparison, we attached a list of top-$10$ database experts generated by the AMiner platform, omitting the origin of either list (AMiner/\qlang{}). 10 out of~10 participants stated that the \qlang{} ranking is more accurate. \footnote{We do not provide the lists here since they contain real researcher names.}
 

 %It turns out that this query is quite useful: AMiner uses a machine learning approach to determine the level of expertise \cite{Tang:2011:TLE:1938275.1938277}. We have sent the two lists of top-$10$ experts by our query and by AMiner's ranking to~$10$ researchers from the field, without explaining how they were created (the overlap of the two lists, ignoring the order, was~60\%). All of the replies preferred our rankings as more accurate.\footnote{We do not provide the lists here, since the personal judgments that we obtained may offend some of the readers. However, the lists can be easily recomputed.}


 %The \qlang{} query in Figure~\ref{fig:experts} selects users who are specialized in databases (or have a similar skill, see the first \textcd{SIMILAR} clause), and who frequently publish at two top conferences in the field, SIGMOD and EDBT (see the second \textcd{SIMILAR} clause). Among the users who match these criteria, the query returns those with the top-$10$ H-index (see the \textcd{ORDER BY} clause, where \textcd{?h} is defined in line~3). 

%As described in Section \ref{sec:related}, much effort is
%made to determining user's level of expertise and finding the system experts. Using \qlang{} queries, one may consider any definition of expert, and select user according to this definition.
%Stack Overflow provides to its users a ranking of top answerers active in a single tag, based on the user's previous scores in answers related to that tag. Naturally, this ranking can be done using simple \qlang{} query.   
%We examine several natural and intuitive strategies for experts finding over AMiner data, and expressed them using \qlang{}. We then compared to Aminer's own experts lists. One example for query is depicted in Figure \ref{fig:experts}.
%The query selects users who are specialized at Database, or some other semantically related area, and in which frequently published at VLDB and SIGMOD, ordered in descending order by their H-index. We Compare our results to AMiner rank of experts in Database. AMiner uses a ML approach 
%for determine the level of expertise \cite{Tang:2011:TLE:1938275.1938277}.
%Our query resulted in same results to AMiner's ranking in $60\%$ of the cases. We then let $10$ researchers in Database to judge the differences between the two list. The majority of researchers pointed our query results were more accurate \footnote{We did not provide the two lists, to avoid conflict of interest.}.  

\textsf{2. Link prediction.} This problem is mostly known from the study of links that are likely to form in a (social) network. The \qlang{} query we have used on the AMiner dataset in Section~\ref{sec:experiments} is one example for such query, which uses hard constraints to select collaborators of collaborators, then employs combined similarity to find the researchers most similar to a given researcher, in terms of their profiles.
% and finally returns the most similar such researchers. 

\begin{figure}
	{\scriptsize
\begin{Verbatim}
SELECT ?u
FROM basic-profile(?u) WHERE
	\{ ?u h-index ?h. \}
SIMILAR basic-profile(?u) TO \{?u keyTerm Databases .\}
	WITH SIMILARITY > 0.5
SIMILAR extended-profile(?u) TO
	\{?u published_at Top\_DB\_Conference. \}
	WITH SIMILARITY > 0.2
ORDER BY DESC(?h)
		\end{Verbatim}
	} \vspace{-6mm} \caption{\qlang{} query selecting database experts.} \label{fig:experts}
	%\vspace{-4mm}
\end{figure} 

\vspace{1px}  



%The well-known problem
%of link prediction, i.e., predicting which links are likely to form in a network is also illustrated in Section \ref{sec:Implementation}.

%\begin{figure}
%	{\scriptsize
%		\begin{Verbatim}
%		SELECT ?u
%		FROM basic-profile(?u) WHERE
%			\{ ?u lastAccessedDate ?d. Filter(?d > 2015) .\}
%		SIMILAR extended-profile(?u) TO
%		\{?u answeredOn Java . ?u answeredOn Python .\}
%			WITH SIMILARITY AS querySim > 0.2
%		ORDER BY querySim
%		\end{Verbatim}
%	} \vspace{-4mm} \caption{\qlang{} query selecting all the Stack Overflow users who have frequently answered Java- and/or %Python-related questions since 2015.} \label{fig:context}
%	%\vspace{-4mm}
%\end{figure}


\textsf{3+4. Profile-based} and \textsf{Context-based} user recommendation aim to provide a given user with recommendations of other \emph{relevant users} from some repository. %\footnote{Do not confuse with similar scenarios in recommending \emph{items} to a given user.}.
E.g., the query in Figure~\ref{fig:qlang} selects appropriate matches to a user on an online dating platform, and in Figure~\ref{fig:SOquery} the query selects users adequate for answering a professional question on a given topic. The former uses \emph{similarity to a target profile} (i.e. the profile of Isabella's ``ideal'' match in Figure~\ref{fig:qlang}). The latter query also takes the \emph{context} into account, by ranking higher candidates that frequently answered other questions in similar topics (lines~6-8 in Figure~\ref{fig:SOquery}). 

 %These recommendations are based, respectively, on \emph{similarity to a target profile}, such as the profile of the given user or of her ``ideal'' match, and \emph{context} such as a set of keywords or properties provided by the user. Context-based recommendations typically also take into account the profile of the given user~\cite{dinoia2012linked,mavridis2015skill}. For example, the queries in Figure~\ref{fig:qlang} and in Figure~\ref{fig:SOquery} include profile-based recommendations since they retrieve users whose profile resembles a target user (Isabella/Basil Bourque). Figure~\ref{fig:context} presents a query that selects SO users that are most relevant to the context of the tags ``Java'' and ``Python'', considering only data from 2015 and on.  
%Context-based user
%recommendation is well studied problem of recommending users, when given a set of keywords is given by a user \cite{cohen2011principles,dinoia2012linked,mavridis2015skill}. The goal is to return a high-quality ranked of users who are relevant both to the keywords and the user asking. An example of this scenario along with the \qlang{} query is illustrated in Section \ref{sec:Implementation}.


\begin{figure}
	{\scriptsize
		\begin{Verbatim}
SELECT ?u
SIMILAR extended-profile(?u) TO
	\{?u collaboratedWith X. 
	 ?u collaboratedWith Y. 
	 ?u collaboratedWith Z. \}
	WITH SIMILARITY AS collaboration
ORDER BY collaboration
		\end{Verbatim}
	} \vspace{-6mm} \caption{\qlang{} query detecting additional community members.} \label{fig:community}
	%\vspace{-4mm}
\end{figure}

\textsf{5. Community Extension.} 
Given a Community in a social network (i.e.\ a set of highly interlinked users), \qlang{} may be used to detect additional  potential members. For example, in Figure \ref{fig:community}, given a seed collaboration community in AMiner comprising of three current members (`X',`Y', and `Z'), the \qlang{} query detects potential members by examining their past collaborations with current members, and orders the retrieved users by their similarity scores, i.e., frequency and relevance of collaborations.

%This is the problem of identifying a highly interlinked community in a network, possibly with respect to some context, e.g., detect all users in a social network that belong to the same nuclear family.
%In some cases, detecting a community having no prior knowledge requires to evaluate the relevance of each member to the others, and hence cannot be done via a single \qlang{} query, since its current semantics allows considering each candidate user only once. 
%However, \qlang{} can be used to detect a community of users who are relevant to some topic (as in context-based recommendation) or to detect additional members of a seed community. 

To conclude, we demonstrated that queries that have a simple structure in \qlang{} can capture various common scenarios of selecting users by their profiles.
% Moreover, we showed that different strategies for user selection are easily and intuitively expressed using \qlang{}. 


\section{Related Work}
\label{sec:related}

\subsection{Diversification}
\label{sec:div}
The diversification of retrieved items, not necessarily users, has been extensively studied in two prominent fields: search and recommender systems. We next overview the most relevant approaches in these fields.

\paragraph*{Diverse Search Results}
%\label{sec:diversesearch}
For years now, \emph{search results diversification} has been  studied in the field of information retrieval (IR)  ~\cite{servajean2013profile,AgrawalGHI09,carbonell1998MMR}. A study ~\cite{vee2008efficientdiv} has shown that a diverse set of search query results has the ability to enhance user satisfaction rate significantly. Search engines which avoid applying diversification techniques on results are more likely to return redundant top-ranked documents and thus fail to answer the often ambiguous ~\cite{anagon2005sampling} user query. Apart from solving query ambiguity, diversification is also used to negate over-personalization of search results~\cite{radlinskiD20006persondiv}.
Drosou et al ~\cite{drosou2010search} proposes the following classification of diversity definitions: \emph{content-based} which characterizes diversity as a $p$-dispersion problem, where the goal is to choose $p$ out of $n$ results in a way that the minimum distance between any pair of chosen points is maximized. \emph{Novelty-based} definitions categorize an item as diverse with respect to a set of past items if it contains novel information. Lastly, \emph{coverage-based} definitions aim to retrieve a set of documents that ``cover'' the many interpretations a certain user's query might have. Also see~\cite{zheng2012coverage} for a survey. In our research, we are inspired by coverage-based diversification, which is particularly suitable when considering selecting representatives from a user repository. In contract with IR methods, which mostly consider the coverage of (possibly overlapping) result categories, user data in our setting has unique characteristics such as many dimensions and the need to ``cover'' ranges of values for each dimension, which must be considered in diversification solutions (Section~\ref{sec:notion}).

\paragraph*{Diverse Recommender Systems}
The goal of recommender systems is to predict the rating a user would assign to a certain item, where items with a high prediction rating are considered most relevant to the client. Diversification in this context, similarly to IR, is used to ensure that the items recommended by the system are also sufficiently different and capture various needs of the client. It is possible to diversify by utilizing the item's semantic data and using some of the methods 
reviewed in the previous paragraph~\cite{servajean2013profile}, but in recent years a \emph{collaborative filtering} (CF) approach ~\cite{su2009cfsurvey}, that relies on the ratings of other users rather than semantic information , is often preferred. Boim et al.~\cite{boim2011priori} suggests using an algorithm based on a modified cover-tree data structure in order to produce diverse CF recommendations. In ~\cite{yu2009variety} the notion of \emph{explanation-based diversity} is presented: in this context, an explanation for the selection of an item may reflect the other similar items the user has rated in the past and/or the similar users which caused a certain item to be selected; then, it is suggested that selecting items with diverse explanations would manifest itself with a diversified recommendation set. In contrast, in our context there are no ready ``explanations'' that can be derived from the user selection process, and then used for aposteriori diversification; but rather the user selection process is inherently driven by coverage and diversification considerations. Moreover, to our knowledge, coverage-based approaches have not been considered in the context of recommender systems.  

\subsection{User Selection}
We next describe prominent contexts in which user selection is considered.

\paragraph*{Expert Finding}
\emph{Expert finding} refers to the act of selecting the fittest users (in terms of relevant skill or knowledge) to perform a  specific task within a large source population of expert users. The field has been a main focus of research within the IR community~\cite{tang2011expertisehetro,campbell2003expertemail}, particularly in the domain of \emph{social networks}~\cite{zhang2007expertsoc,bozzon2013choosing}. Correspondingly, we also propose a method to ensure selection of the most suitable users for our goal, which is to procure a diverse set of opinions. The main difference is that while expert finding focuses on the most capable users, we aim to ``cover'' the full spectrum of available user profiles (e.g., both low and high skill as reflected in the source population). A line of work in expert finding considers the selection of a team or a set of experts. We elaborate on this more relevant line in Section~\ref{sec:diverseUserPrelim} .

\paragraph*{Crowdsourcing}
\emph{Crowdsourcing} is a type of participative online activity in which multiple Web users of varying knowledge, heterogeneity, and number, undertake a task. Typically, such activities involve an overall goal which is achieved by the completion of many micro-tasks by Web users. For instance, the cleaning of a large knowledge base may be achieved by posting many micro-tasks of verifying concrete facts from the knowledge base.  %Freely speaking, crowd-sourcing is the habit of turning to a group of people to obtain needed information and services. Crowd-sourcing is a popular tool used by the average user to accomplish daily tasks such as navigation using the Waze application, travel using the AirBnB platform and KickStarter to start a new business. It is also extremely useful for more advanced users such as computer scientists which turn to Amazon Mechanical Turk to conduct large-scale experiments. The human factor in any crowd-sourcing scenario is crucial, leading researchers to develop several approaches for user selection in a crowd-sourcing environment [todo:ADDREF].
Various studies have considered the filtering of users who undertake a task, by different criteria. This includes, in particular, the assessment of crowd worker skill and filtering low-skill workers~\cite{ipeirotis2010quality}; the filtering of low trust or spammer users~\cite{raykar2012spam}; and the filtering of slow or inefficient users~\cite{haas2015CLAMShell}. These works are orthogonal to ours: we assume all the properties of a user are given, and this may include skill/trust/efficiency metrics derived by automatic tools as the ones mentioned here. 

We also mention here~\cite{amsterdamer2016december}, a declarative tool that allows customized user selection from a repository of user profiles and may be used in the context of crowdsourcing. Our flexible model for user profiles is inspired by this work; however, they do not consider the diversification of selected users.

%\scream{User selection in social networks?}


\subsection{Diverse User Selection}
\label{sec:diverseUserPrelim} 

\paragraph*{Team Formation}
In social networks, \emph{Team Formation} is the task of finding a group of people with the necessary skill-set to perform a given task~\cite{lappas2009finding}. Several articles focus on the importance of minimizing communication cost among team members~\cite{lappas2009finding,kargar2012eff}, which could have a negative effect on the diversity of team members. Maintaining a high level of diversity is important in order to increase creativity which is required to handle complex tasks~\cite{buccafurri2014driving}. Cohen and Yashinski~\cite{cohen2017Crowdiv}, propose an algorithm for team formation with a desired diversity constraint. Diversity is based on user personal properties, similar to our research. However, Cohen and Yashinski tries to make a user selection which distributes as similarly as possible to a pre-defined distribution over user properties, while in our research we do not assume such a target distribution.

\paragraph*{User Selection For Opinions}
%\label{sec:opinionrel}
In Section~\ref{sec:bg} we have presented the many aspects of diverse user selection. The present work is motivated by the procurement of diverse \emph{opinions}, and thus considers the \emph{full range} of scores assigned to any property, accounting for e.g.\ low and high ratings. This is in contrast with the coverage of document topics or expert finding. The recent work of~\cite{wu2015hear} is the most relevant to ours since it also studies diverse opinion procurement. However, they do not consider multi-dimensional data nor customization. %While our approach explicitly relies on a predefined set of properties for the grouping of users, other approaches may attempt to \emph{compute} the ``best'' groups using clustering methods (e.g.~\cite{boim2011diversification}). However, for such approaches, the explanation and refinement of the groups may be highly cumbersome to a client, and thus they are not practical for customization. Previous work has studied customizable user selection in different setting (e.g.,~\cite{amsterdamer2016december,fan2015icrowd}), and is complementary to our present study which focuses on diversification.
\section{CONCLUSION AND FUTURE WORK}
\label{sec:conc}

This work presents \qlang{}, a declarative framework that allows specification of customized user
selection criteria. Its SPARQL-based query language has embedded constructs for capturing the
properties and similarity of (relevant parts of) user profiles, via
a semantic-aware similarity measure. Dedicated algorithm and optimizations 
allow for efficient query processing. Our experiments on
real-life data indicate the effectiveness and usefulness of our approach.


%The queries used in our experiments were manually written and
%reflected the type of users that we intuitively believed to match
%each scenario. An interesting future research direction would be to auto-generate \qlang{} queries for a given task (possibly also described
%declaratively).


% Using \sysname{} query builder interfaces, even novice
% users can perform complex crowd selection queries, then after short
% execution times the results are transfered automatically to the
% hosting crowdsourcing platform.

% While this work mostly focuses on
% crowdsourcing, its approach can be useful for other
% applications, such as social networks or recommender systems, and may
% be employed to refine the set of users/opinions taken
% into consideration.

Interesting directions for future work include  \emph{diversification}, \emph{clustering} and \emph{classification}
constructs, which may be built on top of our semantic notions of
similarity. The ``cold start'' problem of an initially small profile repository can be further considered, possibly by actively asking users for 
missing information or by using external sources. Another interesting research direction is considering the ``gray sheep'' problem \cite{ghazanfar2014leveraging}, advising users how to modify the query to get more/less results. Finally, it would be interesting to adapt our novel similarity measure to other applications, such as entity matching.

% Finally, the automatic generation of \qlang{} queries from
% user questions is another intriguing direction for future research.


%First, auto-generationIn Section~\ref{sec:implementation}, we experimented with auto-generating, task-oriented \qlang{} queries,
%\qlang{} can also be extended to incorporate more features, most importantly to yield \textit{diversified} crowd member results.
%Another interesting extention would be to use \sysname{} for detecting anomalous users such as fake profiles in social networks and spammers in crowdsourcing platforms.
%In Section~\ref{sec:similarity} we shortly discussed the \textit{cold start} problem that may arise when there is insufficient data in the system. It may be interesting to adapt existing solutions e.g. \textit{active learning} by querying users for missing data, using external resources (e.g. profiles in social networks)






%We presented a semantically-rich, similarity metric, embedded in \qlang{} , that naturally captures similarities between crowd profiles and history.

%To allow for efficient query execution, we implement in \sysname{} novel algorithms based on our generic,
%semantically-aware definitions of crowd member similarity and expertise.
%Experimental results with real-life crowd data demonstrate the feasibility and effectiveness of the approach

%including refined preferences for crowd members' attritbutes similarity.

%\sysname{} is based on an RDF data model for representing members' profiles and histories as well as an \textit{ontology} that captures the semantic relations between them.
%To evaluate \qlang{} queries, we developed novel semantic similarity functions based on the \textit{In}




%This tool enables composing and executing queries in a novel, SPARQL-based language, \qlang{}, that captures the desired characteristics of crowd members, including complex constructs such as similarity and expertise, in a generic manner. These queries are executed by \sysname{} over a a repository of the profiles and past answers of crowd members, by performing a semantic analysis of the knowledge provided by each crowd member.






%\paragraph*{Acknowledgments}
%\scream{complete}







% ensure same length columns on last page (might need two sub-sequent latex runs)
\balance



\newpage
% The following two commands are all you need in the
% initial runs of your .tex file to
% produce the bibliography for the citations in your paper.
\bibliographystyle{abbrv}
%\bibliography{vldb_sample}  % vldb_sample.bib is the name of the
\bibliography{sigproc}
% Bibliography in this case
% You must have a proper ".bib" file
%  and remember to run:
% latex bibtex latex latex
% to resolve all references





\end{document}
