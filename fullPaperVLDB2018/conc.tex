\section{CONCLUSION AND FUTURE WORK}
\label{sec:conc}

This work presents \qlang{}, a declarative framework that allows specification of customized user
selection criteria. Its SPARQL-based query language has embedded constructs for capturing the
properties and similarity of (relevant parts of) user profiles, via
a semantic-aware similarity measure. Dedicated algorithm and optimizations 
allow for efficient query processing. Our experiments on
real-life data indicate the effectiveness and usefulness of our approach.


%The queries used in our experiments were manually written and
%reflected the type of users that we intuitively believed to match
%each scenario. An interesting future research direction would be to auto-generate \qlang{} queries for a given task (possibly also described
%declaratively).


% Using \sysname{} query builder interfaces, even novice
% users can perform complex crowd selection queries, then after short
% execution times the results are transfered automatically to the
% hosting crowdsourcing platform.

% While this work mostly focuses on
% crowdsourcing, its approach can be useful for other
% applications, such as social networks or recommender systems, and may
% be employed to refine the set of users/opinions taken
% into consideration.

Interesting directions for future work include  \emph{diversification}, \emph{clustering} and \emph{classification}
constructs, which may be built on top of our semantic notions of
similarity. The ``cold start'' problem of an initially small profile repository can be further considered, possibly by actively asking users for 
missing information or by using external sources. Another interesting research direction is considering the ``gray sheep'' problem \cite{ghazanfar2014leveraging}, advising users how to modify the query to get more/less results. Finally, it would be interesting to adapt our novel similarity measure to other applications, such as entity matching.

% Finally, the automatic generation of \qlang{} queries from
% user questions is another intriguing direction for future research.


%First, auto-generationIn Section~\ref{sec:implementation}, we experimented with auto-generating, task-oriented \qlang{} queries,
%\qlang{} can also be extended to incorporate more features, most importantly to yield \textit{diversified} crowd member results.
%Another interesting extention would be to use \sysname{} for detecting anomalous users such as fake profiles in social networks and spammers in crowdsourcing platforms.
%In Section~\ref{sec:similarity} we shortly discussed the \textit{cold start} problem that may arise when there is insufficient data in the system. It may be interesting to adapt existing solutions e.g. \textit{active learning} by querying users for missing data, using external resources (e.g. profiles in social networks)






%We presented a semantically-rich, similarity metric, embedded in \qlang{} , that naturally captures similarities between crowd profiles and history.

%To allow for efficient query execution, we implement in \sysname{} novel algorithms based on our generic,
%semantically-aware definitions of crowd member similarity and expertise.
%Experimental results with real-life crowd data demonstrate the feasibility and effectiveness of the approach

%including refined preferences for crowd members' attritbutes similarity.

%\sysname{} is based on an RDF data model for representing members' profiles and histories as well as an \textit{ontology} that captures the semantic relations between them.
%To evaluate \qlang{} queries, we developed novel semantic similarity functions based on the \textit{In}




%This tool enables composing and executing queries in a novel, SPARQL-based language, \qlang{}, that captures the desired characteristics of crowd members, including complex constructs such as similarity and expertise, in a generic manner. These queries are executed by \sysname{} over a a repository of the profiles and past answers of crowd members, by performing a semantic analysis of the knowledge provided by each crowd member.






%\paragraph*{Acknowledgments}
%\scream{complete}
